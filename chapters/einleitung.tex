\chapter{Einleitung}
\label{cha:einleitung}

\section{Motivation und Hintergrund}
\label{sec:motivation-und-hintergrund}

\subsection{Allgemeine Motivation}
\label{subsec:allgemeine-motivation}

Die automatisierte Dokumentenverarbeitung zur Gewinnung strukturierter Daten gewinnt zunehmend an Bedeutung. Unternehmen digitalisieren ihre Geschäftsprozesse und nutzen Dokumentenverarbeitungssysteme, um Zeit und Kosten zu sparen. Diese Systeme reduzieren den Bedarf an manueller Dateneingabe und minimieren so das Risiko menschlicher Fehler \cite{PerotVincent2024LLMD}.

Die Industrie wendet dieses Thema praktisch an. Die Firma Multidata, welche selbst ein Hersteller eines Dokumentenverarbeitungssystems, stellt die Aufgabenstellung für diese Bachelorarbeit und unterstreicht damit die Relevanz des Themas.

\subsection{Ausgangslage}
\label{subsec:ausgangslage}

Multidata nutzt aktuell ein Zonal-\gls{OCR}-basiertes\ref{subsubsec:zonal-ocr-parser} System. Benutzer*innen definieren manuell Vorlagen für die Verarbeitung gleichartiger Dokumente. Dieses System erfordert einen hohen manuellen Aufwand. Ein System mit höherem Automatisierungsgrad soll es ersetzen. Kapitel \ref{sec:analyse-des-akutell-eingesetzten-systems} erläutert die Funktionsweise des aktuellen Systems detailliert.

\section{Problemstellung}
\label{sec:problemstellung}

Das angestrebte Dokumentenverarbeitungssystem umfasst zwei Kernaufgaben: Dokumentenkategorisierung und Datenextraktion.

Das System muss zunächst selbstständig die Art des Dokuments bestimmen. Eine dynamische Konfiguration der verfügbaren Kategorien ermöglicht die einfache Erweiterung um neue Dokumententypen ohne umfangreiche Systemänderungen oder erneutes Modelltraining.

Anschließend extrahiert das System die für die Kategorie vorgesehenen Daten. Benutzer*innen definieren die zu extrahierenden Informationen und deren Datentyp. Das System überführt die extrahierten Daten in eine strukturierte Form zur automatisierten Weiterverarbeitung.

Da insbesondere im Umgang mit sensiblen Dokumenten der Datenschutz und die Sicherheit sehr wichtig sind, besteht die Notwendigkeit, das Dokumentenverarbeitungssystem lokal und ohne Übermittlung von Daten an externe Dienste zu betreiben \cite{HuangYupan2022LPfD}. Zusätzlich soll das System plattformunabhängig sein und sowohl auf Windows- als auch auf Linux-Betriebssystemen lauffähig sein, um die Deploymentmöglichkeiten für die Kunden der Firma Multidata zu verbessern.

\section{Zielsetzung und Abgrenzung des Themas}
\label{sec:zielsetzung-und-abgrenzung}

Das Ziel der Arbeit ist es, eine geeignete Methode zur Umsetzung des beschriebenen Dokumentenverarbeitungssystems zu erarbeiten. Dazu sollen verschiedene Ansätze aus wissenschaftlichen Arbeiten verglichen werden. Zusätzlich sollen auch kommerzielle Angebote analysiert und verglichen werden, um ein ausreichend breites Bild des momentanen Stands der Technik zu erhalten.

Es soll erhoben werden, wie zuverlässig die verschiedenen Ansätze sind. Dabei soll sowohl betrachtet werden, ob die benötigten Daten gefunden werden, als auch ob das System falsche Daten erzeugt. Zusätzlich soll auch die Leistung der verschiedenen Ansätze in Bezug auf ihre empirische Laufzeit in die Vergleiche einfließen.

Als Teil dieser Analysen soll auch betrachtet werden, wie sich etwaige Schritte zur Vorverarbeitung der Dokumente auswirken. Schlussendlich soll untersucht werden, wie flexibel die verschiedenen Methoden in Bezug auf Anpassung und Veränderung sind.

Die konkrete Umsetzung im Rahmen eines zur Veröffentlichung fertigen Produktes ist nicht Teil der Zielsetzung.

\section{Methodisches Vorgehen}
\label{sec:methodisches-vorgehen}

Zuerst werden verschiedene Ansätze aus wissenschaftlichen Quellen sowie existierende Software-Bibliotheken und kommerzielle Produkte recherchiert und auf Erfüllung der definierten Anforderungen überprüft. Ansätze, die die Kriterien erfüllen, werden anschließend umgesetzt. Es werden verschiedene Tests zur Bestimmung von Genauigkeit und empirischer Laufzeit durchgeführt. Dazu werden sowohl öffentlich verfügbare Datensätze als auch von Multidata zur Verfügung gestellte Datensätze genutzt. Die Ergebnisse werden anschließend grafisch aufbereitet und verglichen. Abschließend werden verschiedene Verbesserungsansätze, wie unterschiedliche Methoden zum Vorbereiten der Dokumente, implementiert und ihre Auswirkungen gemessen und dokumentiert.